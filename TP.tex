\documentclass[12pt,a4paper]{article}
\usepackage[utf8]{inputenc}
\usepackage[french]{babel}
\usepackage[T1]{fontenc}
\usepackage{amsmath}
\usepackage{amsfonts}
\usepackage{amssymb}
\usepackage{graphicx}
\usepackage{hyperref}
\usepackage{geometry}
\usepackage{listings}
\usepackage{color}

\geometry{margin=2.5cm}

\definecolor{codegray}{rgb}{0.5,0.5,0.5}
\definecolor{codepurple}{rgb}{0.58,0,0.82}
\definecolor{backcolour}{rgb}{0.95,0.95,0.92}

\lstset{
    language=Python,
    backgroundcolor=\color{backcolour},
    commentstyle=\color{codegray},
    keywordstyle=\color{blue},
    numberstyle=\tiny\color{codegray},
    stringstyle=\color{codepurple},
    basicstyle=\ttfamily\small,
    breakatwhitespace=false,
    breaklines=true,
    captionpos=b,
    keepspaces=true,
    numbers=left,
    numbersep=5pt,
    showspaces=false,
    showstringspaces=false,
    showtabs=false,
    tabsize=2
}

\title{Guide de Test et Documentation des Travaux Pratiques \\ Traitement d'Images}
\author{Donfack Pascal}
\date{\today}

\begin{document}

\maketitle

\section{Introduction}
Ce document constitue le guide de test exhaustif pour les travaux pratiques (TP) de traitement d'images. Chaque algorithme a été implémenté avec une rigueur académique, privilégiant la manipulation \textbf{manuelle} des pixels (via des boucles imbriquées) au détriment des fonctions pré-établies des bibliothèques de haut niveau. Cette approche garantit une compréhension profonde des transformations matricielles appliquées aux images.

Un aspect crucial de cette révision est l'intégration systématique de la \textbf{visualisation directe}. Chaque script génère désormais un canevas interactif (via Matplotlib) permettant de comparer instantanément l'image source et le résultat du traitement.

\section{Architecture du Projet}
La structure du projet respecte une organisation modulaire par type de traitement :
\begin{itemize}
    \item \textbf{inputs/} : Répertoire central des images de test (\texttt{img1.png} et \texttt{img2.png}).
    \item \textbf{Chap1/ à Chap7/} : Dossiers thématiques regroupant les scripts Python (\texttt{.py}).
    \item \textbf{outputs/} : Sous-répertoires au sein de chaque chapitre contenant les fichiers sauvegardés après exécution.
\end{itemize}

\section{Protocole de Test}
Pour valider le fonctionnement d'un TP, l'utilisateur doit disposer d'un environnement Python configuré avec les bibliothèques suivantes : \texttt{numpy}, \texttt{pillow}, \texttt{matplotlib}.

\subsection{Exécution}
La commande type pour lancer un TP est la suivante :
\begin{lstlisting}
python ChapX/tpX_nom_du_fichier.py
\end{lstlisting}

À l'exécution, une interface graphique s'affiche. L'utilisateur peut ainsi observer :
\begin{itemize}
    \item L'image originale (souvent convertie en niveaux de gris pour le traitement).
    \item Le résultat de l'algorithme (filtrage, segmentation, transformée, etc.).
    \item Des aides visuelles (histogrammes de fréquence, spectres de magnitude, zones de croissance, etc.).
\end{itemize}

\section{Détail des Chapitres}

\subsection{Chapitre 1 - Fondamentaux}
Ce chapitre couvre les bases : conversion en niveaux de gris, échantillonnage spatial, quantification et profils d'intensité.
\begin{itemize}
    \item \texttt{tp1_rgb_to_gray.py} : Conversion manuelle RGB vers Gris.
    \item \texttt{tp1_gray_quantization.py} : Réduction du nombre de bits.
    \item \texttt{tp1_spatial_sampling.py} : Réduction de la résolution spatiale.
\end{itemize}

\subsection{Chapitre 2 - Traitements Ponctuels}
Calcul d'histogrammes, étirement de contraste, correction gamma et égalisation. Les scripts affichent systématiquement l'histogramme avant/après.
\begin{itemize}
    \item \texttt{tp2_hist_equal.py} : Égalisation globale d'histogramme.
    \item \texttt{tp2_local_hist_equal.py} : Égalisation adaptative locale (fenêtre glissante).
\end{itemize}

\subsection{Chapitre 3 - Filtrage de Convolution}
Implémentation d'un moteur de convolution 2D manuel.
\begin{itemize}
    \item \texttt{tp3_gaussian_filter.py} : Lissage gaussien.
    \item \texttt{tp3_median_filter.py} : Filtre non-linéaire pour bruit impulsionnel.
\end{itemize}

\subsection{Chapitre 4 - Domaine Fréquentiel}
Transformée de Fourier Discrète (DFT). Pour des raisons de performance, la DFT 2D est calculée sur un patch de 64x64.
\begin{itemize}
    \item \texttt{tp4_dft_2d.py} : Visualisation du spectre de magnitude (logarithmique).
\end{itemize}

\subsection{Chapitre 5 - Détection de Contours}
Opérateurs Sobel, Prewitt, et Laplacien.
\begin{itemize}
    \item \texttt{tp5_sobel.py} : Affiche les gradients Gx, Gy et la magnitude finale.
\end{itemize}

\subsection{Chapitre 6 - Segmentation}
Seuillage d'Otsu, croissance de régions et K-means.
\begin{itemize}
    \item \texttt{tp6_otsu_threshold.py} : Calcule le seuil optimal automatiquement.
\end{itemize}

\subsection{Chapitre 7 - Morphologie Mathématique}
Opérations sur images binaires (Érosion, Dilatation, Ouverture, Fermeture).
\begin{itemize}
    \item \texttt{tp7_freeman_chain_code.py} : Suivi de contour par code de Freeman.
\end{itemize}

\section{Conclusion}
Tous les scripts sont conçus pour être simples à lire et à tester. La visualisation directe permet de valider instantanément la correction de la logique implémentée.

\end{document}
